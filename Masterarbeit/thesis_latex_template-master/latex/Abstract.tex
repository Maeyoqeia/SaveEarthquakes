\documentclass[thesis.tex]{subfiles}
\begin{document}

\addchap{Abstract}

wie angekündigt hier die Paper, an die ich gedacht hatte, und nochmal eine Kurzbeschreibung davon, was ich bei Hybridsystem im Kopf hatte.

Das Gerät zeichnet kontinuierlich Wellenformen auf und auf denen müsste man dann Picking machen, also feststellen wann eine seismische Welle auftrifft. Dazu gibt es inzwischen einen ganzen Stapel Paper. Ein guter Einstieg ist Ross et al. (2018) "Generalized Seismic Phase Detection with Deep Learning". Entweder damit gekoppelt (indem man die Trainingsdaten geeignet zusammenstellt) oder als nächster Schritt, muss man unterscheiden, ob es tatsächlich ein Erdbeben oder nur impulsive noise ist. Hier gibt es zum Beispiel Li et al. (2018) "Machine learning seismic wave discrimination: Application to earthquake early warning".

Der übliche Deep Learning Ansatz wäre jetzt Magnitude und Lokalisation direkt zu schätzen, siehe z.B. Mousavi et al. (2019) "A Machine-Learning Approach for Earthquake Magnitude Estimation" und Mousavi et al. (2019) "Bayesian-Deep-Learning Estimation of Earthquake Location from Single-Station Observations". Mein Vorschlag war jetzt, Deep Learning nur für die Distanz zu nutzen und für die Magnitude eine parametrische Modellierung zu wählen. Das ist im early warning eine etablierte Methode, um Magnituden schnell zu schätzen (z.B. Zollo et al. (2006) "Earthquake magnitude estimation from peak amplitudes of very early seismic signals on strong motion records"). Für eine genauere Schätzung könnte man wahrscheinlich Teile der Methode auf Nutzung in Echtzeit adaptieren, die ich hier vorgeschlagen habe (Münchmeyer et al. (2020) "Low uncertainty multifeature magnitude estimation with 3-D corrections and boosting tree regression: application to North Chile"). Eine Echtzeit Magnitudenschätzung wäre schon ein exzellentes Ergebnis und könnte zum Beispiel ziemlich direkt für early warning verwendet werden.

\listoftodos
\end{document}